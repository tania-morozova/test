\documentclass[12pt,titlepage]{article}

\usepackage[cp1251]{inputenc}
\usepackage{amsthm}

\usepackage[russian]{babel}
\usepackage{graphicx}
\usepackage{graphicx,color}
\usepackage{amsfonts}
\usepackage[T2A]{fontenc}
\usepackage{amssymb}
\usepackage{amsthm}

\setlength{\textwidth}{150mm}
\setlength{\textheight}{230mm}
\setlength{\voffset}{-15mm}
\setlength{\hoffset}{-5mm}

\begin{document}


\section{ Постановка задачи }

{\bf{Вариант 11.}}

Задана линейная система обыкновенных дифференциальных уравнений
$$
x'(t) = Ax(t) + Bu(t) + f, \quad t \in [t_{0}, +\infty),
$$
где $ x(t) = (x_{1}(t), \:x_{2}(t))^{T} $ --- двумерный вектор-переменная, $ u(t) = (u_{1}(t), \:u_{2}(t))^{T} $ --- двумерное программное управление, $ A,B \in \mathbb{R}^{2 \times 2} $, $ f \in \mathbb{R}^{2} $. На значения управляющих параметров наложено
ограничение $ u \in P $, где $ P $ --- круг радиуса длины $ r > 0 $ с центром в точке $ p = (p_{1}, p_{2})^{T} $. Пусть $ X_{0} $
и $ X_{1} $ --- соответственно начальное и целевое множества значений фазового вектора, причём $ X_{0} $ --- квадрат со стороной
длины $ k > 0 $ и центром $ x_{0} = (x_{01}, x_{02})^{T} $, $ X_{1} = \{ x_{1} \} $, $ x_{1} = (x_{11}, x_{12})^{T} $. Требуется
решить задачу быстродействия: отыскать наименьшее время $ t_{1} $, в которое фазовая траектория системы, выпущенная в фиксированный момент времени $ t_{0} $ из некоторой точки множества $ X_{0} $, попадает в какую-нибудь точку из $ X_{1} $.

1) Необходимо написать в среде MatLab программу с пользовательским интерфейсом, которая по задаваемым параметрам $ A, B, f, t_{0},
r, p, k, x_{0}, x_{1} $ определяет, разрешима ли поставленная задача быстродействия. В случае разрешимости программа должна (приближённо) найти искомое значение $ t_{1} $, построить графики компонент оптимального управления, компонент соответствующей оптимальной траектории,
сопряжённых переменных (при решении задачи с помощью принципа максимума Понтрягина в линейном случае), а также изображение фазовой траектории на плоскости. В программе не должно быть перебора по $ x_{0} $.

2) В отчёте нужно привести все теоретические выкладки, сделанные в ходе решения задачи оптимального управления, привести нетривиальные примеры построенных оптимальных управлений и траекторий с иллюстрациями. Также требуется посредством приведения подходящего примера
показать разрывность времени оптимального быстродействия $ t_{1} $ по отношению к начальному или целевому множеству фазовых переменных.


\newpage
\section{ Теоретический анализ }

Сформулируем принцип максимума Понтрягина для линейной задачи быстродействия.
Пусть на отрезке времени $ [t_{0}, t_{1}] $ задано некоторое допустимое управление $ u(t) $ такое, что
соответствующее решение $ x(t) $ уравнения
$$
x'(t) = Ax(t) + Bu(t) + f
$$
переводит данный динамический объект из начального множества $ X_{0} $ на конечное множество $ X_{1} $, т.~е. $ x(t_{0}) \in X_{0} $,
$ x(t_{1}) \in X_{1} $, причём $ X_{0} $, $ X_{1} $ выпуклы. Тогда из оптимальности по быстродействию (для которого
момент времени $ t_{0} $ фиксирован, а $ t_{1} $ свободен) допустимого процесса $ (u(t), x(t)) $, $ t \in [t_{0}, t_{1}] $,
следует существование нетривиального решения $ \psi(t) $ сопряжённой системы обыкновенных дифференциальных уравнений
$$
\psi '(t) = -A^{*} \psi(t),
$$
обеспечивающего выполнение таких трёх условий (в нашей задаче третье из них изначально имеет место):
$$
1) \left< Bu(t), \psi(t) \right> = \rho( \psi(t) \,|\, BP ),
$$
$$
2) \left< x(t_{0}), \psi(t_{0}) \right> = \rho( \psi(t_{0}) \,|\, X_{0} ),
$$
$$
3) \left< x(t_{1}), -\psi(t_{1}) \right> = \rho( -\psi(t_{1}) \,|\, X_{1} ).
$$
\normalfont

Для решения поставленной задачи воспользуемся принципом максимума Понтрягина для линейной задачи быстродействия и
так называемой ''функцией цены'' $ \varepsilon[T] $. Сопряжённая к исходной линейной системе обыкновенных дифференциальных уравнений
система принимает вид
$$
\psi '(t) = -A^{*} \psi(t), \quad t \in [t_{0}, +\infty);
$$
её решение для значения $ \psi(t_{1}) $ в произвольной фиксированной точке $ t_{1} > t_{0} $ таково:
$$
\psi(t) = e^{ (t_{1} - t)A^{*} } \psi(t_{1}), \quad t \in [t_{0}, +\infty).
$$
$ \psi(t_{1}) $ будем всегда считать принадлежащим единичной сфере $ S_{1}(0) $ (это законно в силу того, что, каково бы
ни было $ \alpha > 0 $, все три условия принципа максимума не изменятся при замене в них $ \psi $ на $ \alpha \psi $).

Искомые управления мы будем отбирать из класса функций, кусочно непрерывных на соответствующем отрезке задания, вложенном в
$ [t_{0}, +\infty) $. Поэтому условие максимума
$$
\left< Bu(t), \psi(t) \right> = \rho( \psi(t) \,|\, BP ),
$$
преобразуемое к виду
$$
\left< u(t), B^{*}\psi(t) \right> = \rho( B^{*}\psi(t) \,|\, P ),
$$
можно разрешать при всех (а не для почти всех) $ t \in [t_{0}, t_{1}] $. Последнее равенство означает, что $ u(t) $ есть
принадлежащий кругу $ P $ элемент опорного множества к $ P $ в направлении $ B^{*}\psi(t) $. Учтём тот факт, что
всякий вектор, лежащий в круге $ P $, представим в виде суммы вектора $ p $ --- центра $ P $ --- и некоторого вектора
из круга $ S_{r}(0) $ радиуса длины $ r $ с центром в начале координат и что любая такая сумма является элементом $ P $.
Следовательно, $ u(t) - p $ наибольший по норме среди всех векторов из $ S_{r}(0) $, сонаправленных с
$ B^{*}\psi(t) $. В связи с этим имеем
$$
u(t) = p + r \frac{ B^{*}\psi(t) }{ || B^{*}\psi(t) || },
$$
когда $ B^{*}\psi(t) \ne 0 $. Таким образом, теперь мы располагаем явным видом зависимости $ u(t) $ от $ \psi(t_{1}) $.

Время оптимального быстродействия $ t_{1} $ находится (разумеется, приближённо) как наименьший корень, превосходящий $ t_{0} $,
следующего уравнения относительно $ T $:
$$
0 = \varepsilon[T] = \sup_{ ||l|| = 1 } ( <l, x_{1}> - \rho(l \,|\, X[T]) );
$$
здесь через $ X[T] $ обозначено множество достижимости данного линейного динамического объекта, причём соответствующая
опорная функция определяется равенством
$$
\rho(l \,|\, X[T]) = \rho( e^{ (T - t_{0})A^{*} }l \,|\, X_{0} ) + \int_{t_{0}}^{T} \rho( B^{*} e^{ (T - s)A^{*} }l \,|\, P ) ds
  + \int_{t_{0}}^{T} < e^{ (T - s)A }f, l> ds.
$$
При проведении вычислений мы применим такие формулы для опорных функций к множествам $ P $ и $ X_{0} $:
$$
\rho( l \,|\, P ) = <p, l> + r||l||, \quad \rho(l \,|\, X_{0}) = <x_{0}, l> + \frac{k}{2}( |l_{1}| + |l_{2}| ).
$$
Если же у уравнения $ \varepsilon[T] = 0 $ нет корней на интервале $ (t_{0}, +\infty) $, то время оптимального быстродействия
для исходной задачи не существует. Решая задачу на ЭВМ, мы вынуждены выбрать некоторую константу $ t_{max} > t_{0} $,
правее которой корни уравнения $ \varepsilon[T] = 0 $ мы искать не будем. В случае, когда корень из $ (t_{0}, t_{max}] $
не отыскивается численно, программа работу прекращает.

В предлагаемом способе решения задачи проверка выполнения условий 2), 3) из принципа максимума Понтрягина
не нужна, так как они уже предусмотрены в утверждении о том, что $ t_{1} $ есть время оптимального быстродействия
в том и только в том случае, если это наименьший корень уравнения $ \varepsilon[T] = 0 $ из промежутка $ (t_{0}, +\infty) $.

Наконец, предположим, что время оптимального быстродействия $ t_{1} $ существует на отрезке $ [ t_{0}, t_{max} ] $ и подсчитано
нами. Более того, пусть мы нашли какой-либо вектор $ l^{0}(t_{1}) $, доставляющий максимум выражению
$ ( <l, x_{1}> - \rho(l \,|\, X[t_{1}]) ) $ на множестве векторов $ l $ из единичной сферы $ S_{1}(0) $. Он должен являться
значением в точке $ t_{1} $ искомой сопряжённой вектор-функции $ \psi(t) $, которая, стало быть, имеет вид
$$
\psi(t) = e^{ (t_{1} - t)A^{*} } l^{0}(t_{1}), \quad t \in [t_{0}, t_{max}].
$$
По $ \psi(t) $ указанным выше способом определяется управление $ u(t) $, удовлетворяющее условию максимума:
$ u(t) = p + r \frac{ B^{*}\psi(t) }{ || B^{*}\psi(t) || }, B^{*}\psi(t) \ne 0 $. Если $ B^{*}\psi(t) = 0 $,
то $ u(t) $ выбирается произвольно из множества $ P $ (например, полагается $ u(t) = (p_{1} + r, p_{2})^{T} $).
Располагая функцией $ u(t) $, мы можем задать нужную траекторию $ x(t) $ как решение задачи Коши
$$
x'(t) = Ax(t) + Bu(t) + f, \quad t \in [t_{0}, t_{1}], \quad x(t_{1}) = x_{1}.
$$
Теоретически в момент времени $ t_{0} $ последняя обязана находится в точности на границе начального множества $ X_{0} $, но
в ситуации решения задачи на ЭВМ и наличия определённых вычислительных погрешностей это может оказаться не совсем так.

Опишем план программной реализации численного решения задачи. В функциях из файлов AuxFunc1.m и  AuxFunc2.m строится функция,
которая сопоставляет исходным данным и произвольному двумерному вектору $ l $ выражение, максимум которого по
единичной сфере $ S_{1}(0) $ требуется найти для вычисления значения ''функции цены'' $ \varepsilon[T] $.
В файле Epsilon.m приближённо считается значение ''функции цены'' (т.~е. упомянутый максимум) в произвольной точке $ T $,
а в файле L0.m --- координаты соответствующего максимизирующего вектора $ l_{0}(T) $. При этом единичная окружность разбивается
на маленькие дуги одной и той же длины и в граничных точках, разделяющих соседние дуги, сравниваются значения функции
$ \varepsilon $. Таков метод, реализующий ''функцию цены''.

Дальнейшие вычисления совершаются в файле Solve.m. Его содержимое используется в основном файле OC.m. Сначала заданный отрезок
$ [ t_{0}, t_{max} ] $ делится на указанное количество $ N $ малых отрезков одинаковой длины. Поскольку здесь
нужно отыскать наименьший корень уравнения $ \varepsilon[T] = 0 $, то в программе ищется тот из промежуточных
сегментов, на концах которого $ \varepsilon $ принимает значения разных знаков. Если такого отрезка нет, работа
программы завершается. В случае, когда искомый сегмент достигнут, производится его дополнительное разбиение с
целью нахождения более точного приближения времени оптимального быстродействия $ t_{1} $. Следует отметить, что в
программе за $ t_{1} $ берётся правый конец промежуточного сегмента, на концах которого $ \varepsilon $ принимает
значения разных знаков. Действия, которые должны следовать за подсчётом $ t_{1} $, уже разъяснены выше.

Теперь произведём некоторые оценки погрешностей вычисляемых величин. Сначала обратимся ко времени оптимального
быстродействия $ t_{1} $. Исходный отрезок $ [ t_{0}, t_{max} ] $ разбивается на заданное число $ N $ равновеликих отрезков.
Если из них выбран тот, на концах которого $ \varepsilon $ принимает значения разных знаков, то он дополнительно
делится на $ N $ сегментов. Итак, отклонение от искомой величины $ t_{1} $ составляет не более
$ \frac{ t_{max} - t_{0} }{ N^{2} } $.

Далее, при расчёте значения функции $ \varepsilon $ в какой-нибудь момент времени необходимо приближённо найти
такой вектор $ l^{0} \in S_{1}(0) $, на котором достигается соответствующая точная верхняя грань, для чего
единичная сфера разбивается на указываемое в коде программы число $ n $ дуг с одинаковыми длинами. Следовательно,
в силу того, что в треугольнике модуль разности длин двух сторон не превосходит длины третьей стороны и что в
круге длина хорды не больше длины меньшей из соответствующих дуг окружности, погрешность по норме, а потому и
покомпонентная погрешность, в вычислении $ l^{0}(t_{1}) = \psi(t_{1}) $ составляет на более $ \frac{2\pi}{n} $.
В соответствии с этим погрешность по норме и покомпонентная погрешность при подсчёте $ \psi(t) $ в каждой
точке $ t $ взятой временной сетки не больше, чем
$$
|| e^{ (t_{1} - t)A^{*} } || \frac{2\pi}{n}.
$$
Заметим, что функция $ || e^{ (t_{1} - t)A^{*} } || $ переменной $ t $ непрерывна на отрезке $ [ t_{0}, t_{1} ] $
и, стало быть, может быть ограничена на $ [ t_{0}, t_{1} ] $ некоторой положительной постоянной. Управление
$ u(t) $ ввиду представления $ u(t) = p + r \frac{ B^{*}\psi(t) }{ || B^{*}\psi(t) || } $ непрерывно зависит
от $ \psi(t) $.


\newpage
\section{Примеры}

\subsection{}

Здесь
$$
A = \left( \begin{array}{cc}
      -1 & 1 \\
      1 & -2
    \end{array} \right).
$$
Соответствующие собственные значения равны $ \lambda_{1} = \frac{ -3 + \sqrt{5} }{2} $,
$ \lambda_{2} = \frac{ -3 - \sqrt{5} }{2} $, т.~е. положение равновесия представляет собой узел, и притом
устойчивый (ведь $ \lambda_{1} < 0, \lambda_{2} < 0 $). Однако в данной ситуации оптимальная траектория
попадает из квадрата $ X_{0} $ с центром в нуле и стороной длины $ k = 0.2 $ в точку
$ x_{1} = (0.2, -0.7)^{T} $, т.~е. удаляется от начала координат. Это объясняется тем, что норма управления
может принимать достаточно большие значения с границы круга $ P $ радиуса длины $ r = 3 $ с центром
в начале координат, что и происходит в действительности. Таким образом, здесь у $ Bu(t) + f $ влияние
оказывается значительно более значимым, чем у $ Ax(t) $.



В этом примере матрица $ A $ такова:
$$
A = \left( \begin{array}{cc}
      1 & -1 \\
      2 & 1
    \end{array} \right).
$$
Корни её характеристического многочлена равны $ \lambda_{1} = 1 + i\sqrt{2}$, $ \lambda_{2} = 1 - i\sqrt{2} $, её
положение равновесия --- точка $ (0, 0)^{T} $. Это неустойчивый фокус, потому что $ \lambda_{1} $ и
$ \lambda_{2} $ комплексны и имеют положительные вещественные части. Это согласуется с изображением
фазовой траектории, которая, как видно из рисунка, с увеличением времени, направляясь от квадрата $ X_{0} $
к конечной точке $ x_{1} $, отдаляется от нуля. Необходимо подчеркнуть, что в связи с
наличием управления и вектора $ f $ не только матрица $ A $ влияет на качественное поведение системы, но
в силу сказанного выше $ A $ в данном примере играет определяющую роль.



\newpage
\begin{thebibliography}{9}
\bibitem{boltyanskiy} {\it Болтянский В.Г. }{\rm ''Математические методы оптимального управления'', М.: Наука, 1969 г.}
\bibitem{blagodatskykh} {\it Благодатских В.И. }{\rm ''Введение в оптимальное управление. Линейная теория'', М.: Высшая школа, 2001 г.}
\bibitem{govorukhin} {\it Говорухин В.Н., Цибулин В.Г. }{\rm ''Компьютер в математическом исследовании: Maple, MATLAB, LaTeX'', СПб: Питер, 2001 г.}
\end{thebibliography}


\end{document} 