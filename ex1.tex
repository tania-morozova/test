\documentclass[a4paper]{article}

\usepackage[utf8]{inputenc}
\usepackage[english, russian]{babel}
\usepackage[a4paper, left = 2.5cm, right = 1cm, top = 2cm, bottom = 2cm]{geometry}

\usepackage{amsmath}
\usepackage{amsthm}
\usepackage{amssymb}

\linespread{1.58}

\theoremstyle{plain}
\newtheorem{sthm}{Теорема}
\newtheorem{slemm}{Лемма}

\newcommand{\D}{\mathrm{D}}
\newcommand{\E}{\mathrm{E}}
\renewcommand{\le}{\leqslant}
\renewcommand{\ge}{\geqslant}
\renewcommand{\epsilon}{\varepsilon}

\begin{document}


\begin{sthm}[Критерий Коши для равномерной сходимости интеграла, зависящего от параметра]

$I(y)=\int\limits_a^\infty f(x,y)dx $ сходится равномерно на Y $\Leftrightarrow$

$\Leftrightarrow \forall\epsilon>0\ \exists A(\epsilon)\ge a:\ \forall R_1, R_2\ge A(\epsilon),\ \forall y\in Y\ \left|\int\limits_{R_1}^{R_2} f(x,y)dx\right|<\epsilon$.
\end{sthm}
\bigskip
\begin{sthm}[Куранта-Фишера]

Для собственных значений самосопряженного оператора A справедливо представление
$$
\lambda_k=\max_{L_k} \min_{\|{x}\|=1,\ x \in L_k} (Ax,x),
$$
где максимум берется по всевозможным $k$-мерным подпространствам $L_k$ пространства $V$.
\end{sthm}
\bigskip
\begin{slemm}[Грануолла-Беллмана]

Если непрерывная функция $Z(t)$ удовлетворяет условию при $t \ge t_0$
$$
0 \le Z(t) \le k\int\limits_{t_0}^t Z(\tau)d\tau+g(t);\ k=const,
$$
то выполняется оценка
$$
0 \le Z(t) \le k \int\limits_{t_0}^t g(\tau)e^{k(t-\tau)}d\tau + g(t).
$$
\end{slemm}
\bigskip
\begin{sthm}[ЗБЧ в форме Чебышева]

Пусть $X_1, X_2,\ldots, X_n$ --- независимые случайные величины и дисперсия каждой из них существует и ограничена сверху некоторой константой: $\forall i: 1 \le i \le n$ $\exists \D X_i \le C$. Тогда
$$
\forall \epsilon >0 \ \mathrm{P}\left(\left|\dfrac{X_1+\ldots+X_n}{n}-\dfrac{\E X_1+\ldots+\E X_n}{n}\right|<\epsilon\right)\xrightarrow[n\to\infty]{}1.
$$
\end{sthm}
\bigskip
\begin{sthm}[Пойя]
$$
\Phi(t)=P_G(Q(t), Q(t^2),\ldots, Q(t^n)),
$$
где $n=|G|$, $P_G(t_1, t_2, \ldots, t_n)$ --- цикловой индекс группы $G$, а $Q(t^2)$ подставляется в $P_G$ на место переменной $t_k$ ($k=1, 2,\ldots, n$).
\end{sthm}
\end{document}